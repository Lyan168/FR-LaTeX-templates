%%%%%%%%%%%%%%%%%%%%%%% Description du fichier %%%%%%%%%%%%%%%%%%%%%%%
% résumé: Préambule d'un template de prise de notes et de rapports.  %
% auteur: Valérien BRAYE, basé sur le travail de SeniorMars (github) %
%%%%%%%%%%%%%%%%%%%%%%%%%%%%%%%%%%%%%%%%%%%%%%%%%%%%%%%%%%%%%%%%%%%%%%

%%%%%%%%%%%%%%%%%%%%%%%%%%%%%%%%%%%%%%%%%%%%%%%%%%%%%%%%%%%%%%%%%%%%%%%%%%%%%%%
%                               Packages de base                              %
%%%%%%%%%%%%%%%%%%%%%%%%%%%%%%%%%%%%%%%%%%%%%%%%%%%%%%%%%%%%%%%%%%%%%%%%%%%%%%%

% Réduit la taille des marges
\usepackage[a4paper, margin=2cm]{geometry}
% Ajoute plusieurs couleurs.
\usepackage[usenames,dvipsnames,pdftex]{xcolor}
% Permet de change le style des liens.
\usepackage{hyperref}
% Permet d'importer des images et des graphiques.
\usepackage{graphicx}
\graphicspath{{./figures}}
% Permet d'utiliser de nouvelles figures dans des environnements flottants.
\usepackage{float}
% Permet de créer plusieurs colonnes.
\usepackage{multicol}
% Meilleure syntaxe pour les maths.
\usepackage{amsmath,amsfonts,mathtools,amsthm,amssymb}
% Permet de barrer du text.
\usepackage{cancel}
% Permet de modifier le titre des figures.
\usepackage{caption}
% Permet d'importer des pdf directement dans le code LaTeX.
\usepackage{pdfpages}
% Permet d'écrire des algorithmes.
\usepackage[ruled, vlined, linesnumbered]{algorithm2e}
% Change le symbole CQFD (carré blanc en carré noir).
\usepackage{tikzsymbols}
% Traduction en Français des mots générés (date, table des matières...) et caractères français.
\usepackage[french]{babel}
\usepackage[T1]{fontenc}
% Permet l'insertion de texte temporaire
\usepackage{lipsum}
% Corrige les problèmes de taille de police
\usepackage{fix-cm}


% Modifie le symbole CQFD.
\renewcommand\qedsymbol{$\blacksquare$}

\def\class{article}


%%%%%%%%%%%%%%%%%%%%%%%%%%%%%%%%%%%%%%%%%%%%%%%%%%%%%%%%%%%%%%%%%%%%%%%%%%%%%%%
%                              Paramètres de base                             %
%%%%%%%%%%%%%%%%%%%%%%%%%%%%%%%%%%%%%%%%%%%%%%%%%%%%%%%%%%%%%%%%%%%%%%%%%%%%%%%

%%%%%%%%%%%%%%
%  Symboles  %
%%%%%%%%%%%%%%

\let\implies\Rightarrow  % ⟹
\let\impliedby\Leftarrow % ⟸
\let\iff\Leftrightarrow  % ⟺
\let\epsilon\varepsilon  % ε

\AtBeginDocument{\def\labelitemi{$\bullet$}} % change les tirets en points dans les listes
\AtBeginDocument{\def\labelitemii{$\circ$}}

%%%%%%%%%%%%%
%  Figures  %
%%%%%%%%%%%%%

% Paramètres:
% 1. (Optionnel) Largeur de l'image. Défaut: 3/4 de la largeur du texte.
% 2. Chemin du fichier de l'image
% 3. Légende de l'image
\newcommand{\insertfig}[3][0.75]{
  \begin{figure}[H]
    \centering
    \includegraphics[width=#1\textwidth]{#2}
    \caption{#3}
  \end{figure}
}

% Image temporaire
\newcommand{\lipsumimage}[1][Lorem Ipsum]{
  \insertfig{./figures/lipsumimage.png}{#1}
}

%%%%%%%%%%%%
%  Tables  %
%%%%%%%%%%%%

\setlength{\tabcolsep}{5pt}
\renewcommand\arraystretch{1.5}

%%%%%%%%%%%%%%
%  SI Unitx  %
%%%%%%%%%%%%%%

\usepackage{siunitx}
\sisetup{locale = FR}

%%%%%%%%%%
%  TikZ  %
%%%%%%%%%%

\usepackage[framemethod=TikZ]{mdframed}
\usepackage{tikz}
\usepackage{tikz-cd}
\usepackage{tikzsymbols}

\usetikzlibrary{intersections, angles, quotes, calc, positioning}
\usetikzlibrary{arrows.meta}

\tikzset{
  force/.style={thick, {Circle[length=2pt]}-stealth, shorten <=-1pt}
}

%%%%%%%%%%%%%%%
%  PGF Plots  %
%%%%%%%%%%%%%%%

\usepackage{pgfplots}
\pgfplotsset{compat=1.13}

%%%%%%%%%%%%%%%%%%%%%%%%%%%%%%%%%%
%  Centrage de la page de titre  %
%%%%%%%%%%%%%%%%%%%%%%%%%%%%%%%%%%

\usepackage{titling}
\renewcommand\maketitlehooka{\null\mbox{}\vfill}
\renewcommand\maketitlehookd{\vfill\null}

%%%%%%%%%%%%%%%%%%%%%%%%%%%%%%%%%%%%%%%%%%%%%%%%%%
%  Crée un arrière plan gris au milieu d'un PDF  %
%%%%%%%%%%%%%%%%%%%%%%%%%%%%%%%%%%%%%%%%%%%%%%%%%%

\usepackage{eso-pic}
\newcommand\definegraybackground{
  \definecolor{reallylightgray}{HTML}{FAFAFA}
  \AddToShipoutPicture{
    \ifthenelse{\isodd{\thepage}}{
      \AtPageLowerLeft{
        \put(\LenToUnit{\dimexpr\paperwidth-222pt},0){
          \color{reallylightgray}\rule{222pt}{297mm}
        }
      }
    }
    {
      \AtPageLowerLeft{
        \color{reallylightgray}\rule{222pt}{297mm}
      }
    }
  }
}

%%%%%%%%%%%%%%%%%%%%%%%%%%%%%%%%%%
%  Modifie la couleur des liens  %
%%%%%%%%%%%%%%%%%%%%%%%%%%%%%%%%%%

\hypersetup{
  colorlinks,
  linkcolor=blue,
  citecolor={black},
  urlcolor={blue!80!black}
}

%%%%%%%%%%%%%%%%%%
% Fix WrapFigure %
%%%%%%%%%%%%%%%%%%

\newcommand{\wrapfill}{\par\ifnum\value{WF@wrappedlines}>0
    \parskip=0pt
    \addtocounter{WF@wrappedlines}{-1}%
    \null\vspace{\arabic{WF@wrappedlines}\baselineskip}%
    \WFclear
\fi}

%%%%%%%%%%%%%%%%%%
% Multi-Colonnes %
%%%%%%%%%%%%%%%%%%

\let\multicolmulticols\multicols
\let\endmulticolmulticols\endmulticols

\RenewDocumentEnvironment{multicols}{mO{}}
{%
  \ifnum#1=1
    #2%
  \else % Plus qu'une colonne
    \multicolmulticols{#1}[#2]
  \fi
}
{%
  \ifnum#1=1
\else % Plus qu'une colonne
  \endmulticolmulticols
\fi
}

\newlength{\thickarrayrulewidth}
\setlength{\thickarrayrulewidth}{5\arrayrulewidth}


%%%%%%%%%%%%%%%%%%%%%%%%%%%%%%%%%%%%%%%%%%%%%%%%%%%%%%%%%%%%%%%%%%%%%%%%%%%
%                     Commandes spécifiques à l'école                     %
%%%%%%%%%%%%%%%%%%%%%%%%%%%%%%%%%%%%%%%%%%%%%%%%%%%%%%%%%%%%%%%%%%%%%%%%%%%

%%%%%%%%%%%%%%%%%%%%%%%%%%%%%%%%%%%%
%  Initialise le nouveau compteur  %
%%%%%%%%%%%%%%%%%%%%%%%%%%%%%%%%%%%%

\newcounter{lecturecounter}

%%%%%%%%%%%%%%%%%%%%%%%%%
%  NewCommandes utiles  %
%%%%%%%%%%%%%%%%%%%%%%%%%

\makeatletter

\newcommand\resetcounters{
  % Réinitialise les compteurs de sous section, sous sous section et la
  % définition de tous les environnements personnalisés.
  \setcounter{subsection}{0}
  \setcounter{subsubsection}{0}
  \setcounter{paragraph}{0}
  \setcounter{subparagraph}{0}
  \setcounter{theorem}{0}
  \setcounter{claim}{0}
  \setcounter{corollary}{0}
  \setcounter{lemma}{0}
  \setcounter{exercise}{0}

  \@ifclasswith\class{nocolor}{
    \setcounter{definition}{0}
  }{}
}

%%%%%%%%%%%%%%%%%%%%%%%%%%%
%  Commandes de chapitre  %
%%%%%%%%%%%%%%%%%%%%%%%%%%%

\usepackage{xifthen}

% Paramètres:
% 1. (Optionnel) Numéro du chapitre.
% 2. Date et heure du chapitre.
% 3. Titre du chapitre.
% EXEMPLE:
% 1. \lesson{Date (Heure)}{Titre du chapitre}
% 2. \lesson[4]{Date (Heure)}{Titre du chapitre}
% 3. \lesson{Date (Titre du chapitre)}{}
% 4. \lesson[4]{Date (Heure)}{}
\def\@lesson{}
\newcommand\lesson[3][\arabic{lecturecounter}]{
  % Incrémente le lecturecounter.
  \addtocounter{lecturecounter}{1}

  % Met le numéro de section au numéro de chapitre.
  \setcounter{section}{#1}
  \renewcommand\thesubsection{#1.\arabic{subsection}}

  % Réinitialise les compteurs.
  \resetcounters

  % Vérifie si l'utilisateur a fourni le titre du chapitre.
  \ifthenelse{\isempty{#3}}{
    \def\@lesson{Chapitre \arabic{lecturecounter}}
  }{
    \def\@lesson{Chapitre \arabic{lecturecounter}: #3}
  }

  % Affiche l'information sous la forme suivante:
  %                                                                Date (Heure)
  % ---------------------------------------------------------------------------
  % Chapitre 1: Titre du chapitre
  \hfill\small{#2}
  \hrule
  \vspace*{-0.3cm}
  \section*{\@lesson}
  \addcontentsline{toc}{section}{\@lesson}
}

%%%%%%%%%%%%%%%%%%%%%%%%
%  Import des figures  %
%%%%%%%%%%%%%%%%%%%%%%%%

\usepackage{import}
\pdfminorversion=7

% Paramètres:
% 1. Le nom de la figure. Elle doit être dans figures/NOM.tex_pdf.
% 2. (Optionnel) La largeur de la figure. Exemple: 0.5, 0.35.
% EXEMPLE:
% 1. \incfig{limit-graph}
% 2. \incfig[0.4]{limit-graph}
\newcommand\incfig[2][1]{%
  \def\svgwidth{#1\columnwidth}
  \import{./figures/}{#2.pdf_tex}
}

\begingroup\expandafter\expandafter\expandafter\endgroup
\expandafter\ifx\csname pdfsuppresswarningpagegroup\endcsname\relax
\else
  \pdfsuppresswarningpagegroup=1\relax
\fi

%%%%%%%%%%%%%%%%%
% Fancy Headers %
%%%%%%%%%%%%%%%%%

\usepackage{fancyhdr}

% Force une nouvelle page.
\newcommand\forcenewpage{\clearpage\mbox{~}\clearpage\newpage}

% Cette commande simplifie la gestion des entêtes et pieds de page.
\newcommand\createintro{
  % Utilise les chiffres romains (ie. i, v, vi, x, ...).
  \pagenumbering{roman}

  % Affiche le style de la page.
  \maketitle
  % Rend le titre du pagestyle vide, sans fancy headers / footers.
  \thispagestyle{empty}
  % Saute une page
  \newpage

  % Si le fichier "intro.tex" existe, l'ajoute.
  \IfFileExists{intro.tex}{ % si le fichier existe.
    % Ajoute le fichier.
    % TODO: Compléter les textes entre crochets
Ce document contient les notes du cours [intitulé du cours], donné par [nom de l'enseignant] à \faculty, au semestre \term. \\
Ce cours traite de [bref résumé du contenu du cours]. \\
Tout le contenu de ce document revient au professeur [nom de l'enseignant], tandis que la prise en note et la mise en forme me reviennent.
\begin{remark}
    Ce document contient inévitablement quelques erreurs, simples
    fautes de frappes ou erreurs sur le contenu. Merci de garder en tête que ce ne sont que les notes de cours d'un étudiant en cours de formation, qui ne maîtrise pas encore toute la matière, et de conserver un esprit critique lors de leur lecture. Si vous découvrez des erreurs, quelles qu'elles soient, ou que vous souhaitez ajouter ou modifier du contenu, sentez vous libre de me joindre à l'adresse ci-dessous.
\end{remark}
Me contacter:
\href{mailto:[adresse email]}{[adresse email]}.

    % Rend le pagestyle fancy pour la page intro.tex.
    \pagestyle{fancy}

    % Retire la ligne de l'entête
    \renewcommand\headrulewidth{0pt}

    % Retire tout ce qui est lié à l'entête.
    \fancyhead{}

    % Ajoute ce qui est lié à l'entête au centre.
    \fancyfoot[C]{
      \vspace{-2cm} % remonte le pied de page (sinon marge trop petite)
      \hrule
      \vspace{0.1cm}
      \@author \\
      \term \space \academicyear \\
      Dernière mise à jour: \@date \\
      \faculty
    }
  }

  % Saute une page.
  \newpage

  % Comme précédemment, enlève ce qui est lié à l'entête et le remplace par
  % le numéro de page, toujours en chiffres romains.
  \fancyfoot[C]{\thepage}
  % Ajoute la table des matières.
  \tableofcontents
  \newpage

  % Change la numérotation des pages en chiffres arabes.
  \pagenumbering{arabic}
  % Règle le numéro de page à 1.
  \setcounter{page}{1}

  % Ajoute la ligne de l'entête précédemment enlevée.
  \renewcommand\headrulewidth{0.4pt}
  % Ajoute le titre du chapitre en haut à droite.
  \fancyhead[R]{\@lesson}
  % Ajoute le titre de l'auteur en haut à gauche.
  \fancyhead[L]{\@author}
  % Ajoute le numéro de page en bas au centre.
  \fancyfoot[C]{\thepage}
}

\makeatother


%%%%%%%%%%%%%%%%%%%%%%%%%%%%%%%%%%%%%%%%%%%%%%%%%%%%%%%%%%%%%%%%%%%%%%%%%%%%%%%%%%
%                            Commandes personnalisées                            %
%%%%%%%%%%%%%%%%%%%%%%%%%%%%%%%%%%%%%%%%%%%%%%%%%%%%%%%%%%%%%%%%%%%%%%%%%%%%%%%%%%

%%%%%%%%%%%%%%
%  Encercle  %
%%%%%%%%%%%%%%

\newcommand*\circled[1]{\tikz[baseline=(char.base)]{
  \node[shape=circle,draw,inner sep=1pt] (char) {#1};}
}

%%%%%%%%%%%%%%%%%%%%
%  Commandes TODO  %
%%%%%%%%%%%%%%%%%%%%

\usepackage{xargs}
\usepackage[colorinlistoftodos]{todonotes}

\makeatletter

\@ifclasswith\class{working}{
  \newcommandx\unsure[2][1=]{\todo[linecolor=red,backgroundcolor=red!25,bordercolor=red,#1]{#2}}
  \newcommandx\change[2][1=]{\todo[linecolor=blue,backgroundcolor=blue!25,bordercolor=blue,#1]{#2}}
  \newcommandx\info[2][1=]{\todo[linecolor=OliveGreen,backgroundcolor=OliveGreen!25,bordercolor=OliveGreen,#1]{#2}}
  \newcommandx\improvement[2][1=]{\todo[linecolor=Plum,backgroundcolor=Plum!25,bordercolor=Plum,#1]{#2}}

  \newcommand\listnotes{
    \newpage
    \listoftodos[Notes]
  }
}{
  \newcommandx\unsure[2][1=]{}
  \newcommandx\change[2][1=]{}
  \newcommandx\info[2][1=]{}
  \newcommandx\improvement[2][1=]{}

  \newcommand\listnotes{}
}

\makeatother

%%%%%%%%%%%%%
%  Correct  %
%%%%%%%%%%%%%

% Paramètres:
% 1. L'affirmation incorrecte.
% 2. L'affirmation correcte.
% EXEMPLE:
% 1. \correct{INCORRECT}{CORRECT}
\definecolor{correct}{HTML}{009900}
\newcommand\correct[2]{{\color{red}{#1 }}\ensuremath{\to}{\color{correct}{ #2}}}


%%%%%%%%%%%%%%%%%%%%%%%%%%%%%%%%%%%%%%%%%%%%%%%%%%%%%%%%%%%%%%%%%%%%%%%%%%%%%%%
%                                Environnements                               %
%%%%%%%%%%%%%%%%%%%%%%%%%%%%%%%%%%%%%%%%%%%%%%%%%%%%%%%%%%%%%%%%%%%%%%%%%%%%%%%

\usepackage{varwidth}
\usepackage{thmtools}
\usepackage[most,many,breakable]{tcolorbox}

\tcbuselibrary{theorems,skins,hooks}
\usetikzlibrary{arrows,calc,shadows.blur}

%%%%%%%%%%%%%%%%%%%%%%%%%%%%%%
%  Définitions des couleurs  %
%%%%%%%%%%%%%%%%%%%%%%%%%%%%%%

\definecolor{myblue}{RGB}{45, 111, 177}
\definecolor{mygreen}{RGB}{56, 140, 70}
\definecolor{myred}{RGB}{199, 68, 64}
\definecolor{mypurple}{RGB}{197, 92, 212}

\definecolor{definition}{HTML}{228b22}
\definecolor{theorem}{HTML}{00007B}
\definecolor{example}{HTML}{2A7F7F}
\definecolor{definition}{HTML}{228b22}
\definecolor{prop}{HTML}{191971}
\definecolor{lemma}{HTML}{983b0f}
\definecolor{exercise}{HTML}{88D6D1}

\colorlet{definition}{mygreen!85!black}
\colorlet{claim}{mygreen!85!black}
\colorlet{corollary}{mypurple!85!black}
\colorlet{proof}{theorem}

%%%%%%%%%%%%%%%%%%%%%%%%%%%%%%%%%%%%%%%%%%%%%%%%%%%%%%%%%%%%%%%
%  Crée des environnements de style basés sur les paramètres  %
%%%%%%%%%%%%%%%%%%%%%%%%%%%%%%%%%%%%%%%%%%%%%%%%%%%%%%%%%%%%%%%

\mdfsetup{skipabove=1em,skipbelow=0em}

%%%%%%%%%%%%%%%%%%%%%%
%  Commandes utiles  %
%%%%%%%%%%%%%%%%%%%%%%

% Paramètres:
% 1. Nom du théorème.
% 2. N'importe quels paramètres supplémentaires pour declaretheoremstyle.
% 3. N'importe quels paramètres supplémentaires pour mdframed.
% EXEMPLE:
% 1. \createnewtheoremstyle{thmdefinitionbox}{}{}
% 2. \createnewtheoremstyle{thmtheorembox}{}{}
% 3. \createnewtheoremstyle{thmproofbox}{qed=\qedsymbol}{
%       rightline=false, topline=false, bottomline=false
%    }
\newcommand\createnewtheoremstyle[3]{
  \declaretheoremstyle[
  headfont=\bfseries\sffamily, bodyfont=\normalfont, #2,
  mdframed={
    #3,
  },
  ]{#1}
}

% Paramètres:
% 1. Nom du théorème.
% 2. Color of theorem.
% 2. N'importe quels paramètres supplémentaires pour declaretheoremstyle.
% 3. N'importe quels paramètres supplémentaires pour mdframed.
% EXEMPLE:
% 1. \createnewcoloredtheoremstyle{thmdefinitionbox}{definition}{}{}
% 2. \createnewcoloredtheoremstyle{thmexamplebox}{example}{}{
%       rightline=true, leftline=true, topline=true, bottomline=true
%     }
% 3. \createnewcoloredtheoremstyle{thmproofbox}{proof}{qed=\qedsymbol}{backgroundcolor=white}
\newcommand\createnewcoloredtheoremstyle[4]{
  \declaretheoremstyle[
  headfont=\bfseries\sffamily\color{#2}, bodyfont=\normalfont, #3,
  mdframed={
    linewidth=2pt,
    rightline=false, leftline=true, topline=false, bottomline=false,
    linecolor=#2, backgroundcolor=#2!5, #4,
  },
  ]{#1}
}

%%%%%%%%%%%%%%%%%%%%%%%%%%%%%%%%%%%%%%
%  Crée les environnements de style  %
%%%%%%%%%%%%%%%%%%%%%%%%%%%%%%%%%%%%%%

\makeatletter
\@ifclasswith\class{nocolor}{
  % Environnements sans couleurs.

  \createnewtheoremstyle{thmdefinitionbox}{}{}
  \createnewtheoremstyle{thmtheorembox}{}{}
  \createnewtheoremstyle{thmexamplebox}{}{}
  \createnewtheoremstyle{thmclaimbox}{}{}
  \createnewtheoremstyle{thmcorollarybox}{}{}
  \createnewtheoremstyle{thmpropbox}{}{}
  \createnewtheoremstyle{thmlemmabox}{}{}
  \createnewtheoremstyle{thmexercisebox}{}{}
  \createnewtheoremstyle{thmdefinitionbox}{}{}
  \createnewtheoremstyle{thmquestionbox}{}{}
  \createnewtheoremstyle{thmsolutionbox}{}{}

  \createnewtheoremstyle{thmproofbox}{qed=\qedsymbol}{}
  \createnewtheoremstyle{thmexplanationbox}{}{}
}{
  % Environnements avec couleurs.

  \createnewcoloredtheoremstyle{thmdefinitionbox}{definition}{}{}
  \createnewcoloredtheoremstyle{thmtheorembox}{theorem}{}{}
  \createnewcoloredtheoremstyle{thmexamplebox}{example}{}{
    rightline=true, leftline=true, topline=true, bottomline=true
  }
  \createnewcoloredtheoremstyle{thmclaimbox}{claim}{}{}
  \createnewcoloredtheoremstyle{thmcorollarybox}{corollary}{}{}
  \createnewcoloredtheoremstyle{thmpropbox}{prop}{}{}
  \createnewcoloredtheoremstyle{thmlemmabox}{lemma}{}{}
  \createnewcoloredtheoremstyle{thmexercisebox}{exercise}{}{}

  \createnewcoloredtheoremstyle{thmproofbox}{proof}{qed=\qedsymbol}{backgroundcolor=white}
  \createnewcoloredtheoremstyle{thmexplanationbox}{example}{qed=\qedsymbol}{backgroundcolor=white}
}
\makeatother

%%%%%%%%%%%%%%%%%%%%%%%%%%%%%
%  Crée les environnements  %
%%%%%%%%%%%%%%%%%%%%%%%%%%%%%

\declaretheorem[numberwithin=section, style=thmtheorembox,     name=Théorème]{theorem}
\declaretheorem[numbered=no,          style=thmexamplebox,     name=Exemple]{example}
\declaretheorem[numberwithin=section, style=thmclaimbox,       name=Affirmation]{claim}
\declaretheorem[numberwithin=section, style=thmcorollarybox,   name=Corollaire]{corollary}
\declaretheorem[numberwithin=section, style=thmpropbox,        name=Proposition]{prop}
\declaretheorem[numberwithin=section, style=thmlemmabox,       name=Lemme]{lemma}
\declaretheorem[numberwithin=section, style=thmexercisebox,    name=Exercice]{exercise}
\declaretheorem[numbered=no,          style=thmproofbox,       name=Preuve]{replacementproof}
\declaretheorem[numbered=no,          style=thmexplanationbox, name=Preuve]{expl}

\makeatletter
\@ifclasswith\class{nocolor}{
  % Environnements sans couleurs.

  \newtheorem*{note}{Note}

  \declaretheorem[numberwithin=section, style=thmdefinitionbox, name=Definition]{definition}
  \declaretheorem[numberwithin=section, style=thmquestionbox,   name=Question]{question}
  \declaretheorem[numberwithin=section, style=thmsolutionbox,   name=Solution]{solution}
}{
  % Environnements avec couleurs.

  \newtcbtheorem[number within=section]{Definition}{Definition}{
    enhanced,
    before skip=2mm,
    after skip=2mm,
    colback=red!5,
    colframe=red!80!black,
    colbacktitle=red!75!black,
    boxrule=0.5mm,
    attach boxed title to top left={
      xshift=1cm,
      yshift*=1mm-\tcboxedtitleheight
    },
    varwidth boxed title*=-3cm,
    boxed title style={
      interior engine=empty,
      frame code={
        \path[fill=tcbcolback]
        ([yshift=-1mm,xshift=-1mm]frame.north west)
        arc[start angle=0,end angle=180,radius=1mm]
        ([yshift=-1mm,xshift=1mm]frame.north east)
        arc[start angle=180,end angle=0,radius=1mm];
        \path[left color=tcbcolback!60!black,right color=tcbcolback!60!black,
        middle color=tcbcolback!80!black]
        ([xshift=-2mm]frame.north west) -- ([xshift=2mm]frame.north east)
        [rounded corners=1mm]-- ([xshift=1mm,yshift=-1mm]frame.north east)
        -- (frame.south east) -- (frame.south west)
        -- ([xshift=-1mm,yshift=-1mm]frame.north west)
        [sharp corners]-- cycle;
      },
    },
    fonttitle=\bfseries,
    title={#2},
    #1
  }{def}

  \NewDocumentEnvironment{definition}{O{}O{}}
    {\begin{Definition}{#1}{#2}}{\end{Definition}}

  \newtcolorbox{note}[1][]{%
    enhanced jigsaw,
    colback=gray!20!white,%
    colframe=gray!80!black,
    size=small,
    boxrule=1pt,
    title=\textbf{Note:},
    halign title=flush center,
    coltitle=black,
    breakable,
    drop shadow=black!50!white,
    attach boxed title to top left={xshift=1cm,yshift=-\tcboxedtitleheight/2,yshifttext=-\tcboxedtitleheight/2},
    minipage boxed title=1.5cm,
    boxed title style={%
      colback=white,
      size=fbox,
      boxrule=1pt,
      boxsep=2pt,
      underlay={%
        \coordinate (dotA) at ($(interior.west) + (-0.5pt,0)$);
        \coordinate (dotB) at ($(interior.east) + (0.5pt,0)$);
        \begin{scope}
          \clip (interior.north west) rectangle ([xshift=3ex]interior.east);
          \filldraw [white, blur shadow={shadow opacity=60, shadow yshift=-.75ex}, rounded corners=2pt] (interior.north west) rectangle (interior.south east);
        \end{scope}
        \begin{scope}[gray!80!black]
          \fill (dotA) circle (2pt);
          \fill (dotB) circle (2pt);
        \end{scope}
      },
    },
    #1,
  }

  \newtcbtheorem{Question}{Question}{enhanced,
    breakable,
    colback=white,
    colframe=myblue!80!black,
    attach boxed title to top left={yshift*=-\tcboxedtitleheight},
    fonttitle=\bfseries,
    title=\textbf{Question:},
    boxed title size=title,
    boxed title style={
      sharp corners,
      rounded corners=northwest,
      colback=tcbcolframe,
      boxrule=0pt,
    },
    underlay boxed title={
      \path[fill=tcbcolframe] (title.south west)--(title.south east)
      to[out=0, in=180] ([xshift=5mm]title.east)--
      (title.center-|frame.east)
      [rounded corners=\kvtcb@arc] |-
      (frame.north) -| cycle;
    },
    #1
  }{def}

  \NewDocumentEnvironment{question}{O{}O{}}
  {\begin{Question}{#1}{#2}}{\end{Question}}

  \newtcolorbox{Solution}{enhanced,
    breakable,
    colback=white,
    colframe=mygreen!80!black,
    attach boxed title to top left={yshift*=-\tcboxedtitleheight},
    title=\textbf{Solution:},
    boxed title size=title,
    boxed title style={
      sharp corners,
      rounded corners=northwest,
      colback=tcbcolframe,
      boxrule=0pt,
    },
    underlay boxed title={
      \path[fill=tcbcolframe] (title.south west)--(title.south east)
      to[out=0, in=180] ([xshift=5mm]title.east)--
      (title.center-|frame.east)
      [rounded corners=\kvtcb@arc] |-
      (frame.north) -| cycle;
    },
  }

  \NewDocumentEnvironment{solution}{O{}O{}}
  {\vspace{-10pt}\begin{Solution}{#1}{#2}}{\end{Solution}}
}
\makeatother

%%%%%%%%%%%%%%%%%%%%%%%%%%%%%%%%%%%%%%%%
%  Environnements d'édition de preuve  %
%%%%%%%%%%%%%%%%%%%%%%%%%%%%%%%%%%%%%%%%

\renewenvironment{proof}[1][\proofname]{\vspace{-10pt}\begin{replacementproof}}{\end{replacementproof}}
\newenvironment{explanation}[1][\proofname]{\vspace{-10pt}\begin{expl}}{\end{expl}}

\theoremstyle{definition}

\newtheorem*{notation}{Notation}
\newtheorem*{previouslyseen}{Comme vu auparavant}
\newtheorem*{problem}{Problème}
\newtheorem*{observe}{Obserons que}
\newtheorem*{property}{Propriété}
\newtheorem*{intuition}{Intuition}
\newtheorem*{remark}{Remarque}
